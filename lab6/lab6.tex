\documentclass[11pt,a4paper,oneside, titlepage,reqno]{amsproc}
\usepackage{changepage}
\usepackage{mathtext} % русские буквы в формулах
\usepackage[T2A]{fontenc}
\usepackage[utf8x]{inputenc}
\usepackage{ucs}
\usepackage{cmap}
\usepackage[english,russian]{babel}
\usepackage{graphicx}
%\usepackage{concrete}
%\usepackage{amsmath}
%\usepackage{amsfonts}
\usepackage{amssymb}
\usepackage{dcolumn}
\usepackage{booktabs}
\usepackage{ctable}
\usepackage{multirow}

\newcommand{\specialcell}[2][c]{%
  \begin{tabular}[#1]{@{}c@{}}#2\end{tabular}}
\oddsidemargin = 0pt
\textwidth = 14 cm
\topmargin = -2 cm
\textheight = 24 cm
\makeatletter
\renewcommand{\theequation}{\thesection.\arabic{equation}}
\@addtoreset{equation}{section}
\newcommand{\eq}{\begin{equation}}
\newcommand{\eeq}{\end{equation}}
\newcommand{\fr}{\frac}
\newcommand{\mf}{\mathfrak}
\newcommand{\sub}{\subsection}
\newcommand{\subsub}{\subsubsection}
\newcommand{\definition}{\theoremstyle{definition}}
\newcommand{\mult}[2]{\genfrac{\left[}{\right.}{0pt}{}{#1}{#2}}
\renewcommand{\qed}{\begin{center} $\mathsf{QED}$ \end{center}}
\newcommand{\al}{\alpha}
\newcommand{\comment}[1]{\marginpar{\Small{{\sl #1}}} }
\newcommand{\epigraph}[2]{\begin{flushright} {\em #1}\\#2\\[20 pt]
\end{flushright}}
\newcommand{\fx}[1]{\ensuremath{\mathit{f}_{#1}(x)}}
\newcommand{\re}[1]{(\ref{#1})}
\newcommand{\mh}{\mathit}
\newcommand{\itm}[1]{\begin{itemize}	\item #1 \end{itemize}}
\newcommand{\note}[1]{\begin{flushleft}\hbox{%
\vrule\hspace{.5em}\parbox{ .9\textwidth}%
{ #1}} \end{flushleft}}
\newcommand{\Al}{\ensuremath{\mathcal{A}}}
\newcommand{\@dotsep}{3.9}
\newcommand{\system}[1]{\eq\left\{ \begin{aligned} #1
\end{aligned}\right.\\[5 pt]\eeq}
\renewcommand{\phi}{\varphi}

% http://www.texnik.de/floats/caption.phtml
% This does spacing around caption.
%\setlength{\abovecaptionskip}{6pt}   % 0.5cm as an example
%\setlength{\belowcaptionskip}{9pt}   % 0.5cm as an example

%%%%%%%%%%%%%%%%%%%%%%%%%%%%%%%%%%%%%%%%%%%%%%%%%%%%%%%%%%%%%%%%%%%%%%%

\author{Соколовский Роман}
\begin{document}
\begin{titlepage}
\begin{center}
% Upper part of the page
\textsc{\LARGE ГУАП}\\[2cm]
\textsc{\LARGE Кафедра антенн и эксплуатации радиоэлектронной 
 аппаратуры}
\\[1cm]


 \begin{flushleft} \large
  \emph{Отчёт} \\
   \emph{защищен с оценкой}
   \\[0.5cm]
   \emph{Преподаватель}\\[-4mm]
   \HRule\\[-4mm]
\end{flushleft}
\begin{flushright}
Должность, уч. степень, звание \ \ \ \ \ \ \ подпись, дата \ \ \ \ \ \ \ инициалы, фамилия \\[10mm]
\end{flushright}

\textsc{\Large Отчёт о лабораторной работе №5:} \\[1cm]
\textsc{\Large ИССЛЕДОВАНИЕ ПОВЕРХНОСТЫХ ВОЛН, РАСПРОСТРАНЯЮЩИХСЯ ВДОЛЬ ПЛОСКИХ ЗАМЕДЛЯЮЩИХ СИСТЕМ.}\\[1.5cm]
% Title

% Author and supervisor

\begin{flushleft} \large
\emph{Работу выполнил:}\\
Студент\\ гр. 5025.\\[-4mm]
\HRule\\[-4mm]
\end{flushleft}
\begin{flushright}
подпись, дата \ \ \ \ \ \ \ инициалы, фамилия \\[10mm]
\end{flushright}

\vfill
% Bottom of the page
Санкт-Петербург, 2012
\end{center}
\end{titlepage}

\section{Цель Работы}
\begin{itemize}
    \item Изучение законов отражения плоских электромагнитных волн от плоской проводящей
        поверхности
    \item Изучение структуры поля при нормальном и наклонном падении параллельно
        поляризованной волны на плоскую проводящую поверхность
    \item Изучение структуры поля при наклонном падении перпендикулярно поляризованной волны
        на плоскую проводящую поверхность
    \item Исследование распределения по нормали к экрану амплитуд составляющих электрического
        поля в зависимости от угла падения на проводящий экран параллельно поляризованной
        плоской электромагнитной волны
    \item Исследование волны, направляемой металлической границей раздела
    \item Исследование распределения амплитуж составляющих поля по нормали к экрану в
        зависимости от угла падения на проводящий экран параллельно и перпендикулярно
        поляризованных плоских электромагнитных волн.\\
\end{itemize}

\section{Схема Лабораторной Установки}
Схема лабораторной установки представлена на Рис. \ref{fig:scheme},
компонеты установки обозначены следующим образом:
\begin{enumerate}
    \item СВЧ-генератор
    \item излучающий пирамидальный рупор
    \item волновод прямоугольного сечения
    \item коаксиальный волновой переход
    \item полуволновой симметричный вибратор
    \item коаксиальный соединитель
    \item детекторная секция
    \item измерительный усилитель
    \item металлический стол с крестообразными прорезями
    \item плоский основной алюминиевый экран
    \item плоский дополнительный алюминиевый экран.
\end{enumerate}

\begin{figure}[h!]
    \begin{center}
        \includegraphics[width=0.60\textwidth]{scheme.jpg}
    \end{center}
    \vspace{-20pt}
    \caption{Принципиальная схема лабораторной установки}
    \label{fig:scheme}
\end{figure}

\newpage
\section{Результаты измерений и вычислений}
\subsection{Измерения и вычисления}
\subsubsection{$\Delta\phi_{расч}$}

\subsubsection{$\Delta\phi_{изм}$}

\subsubsection{Коэффициент эллиптичности без учета различного затухания составляющих вектора}

\subsection{Таблицы результатов измерений и вычислений}
Результаты исследования линейно поляризованной волны приведены в таблице
\ref{tab:tab1}. Полученные характеристики эллиптически поляризованных
волн сведены в эту же таблицу для компактности и удобства.
Графы таблиц, дублирующие таблицы протокола измерений (см. Приложение 1),
здесь приведены не будут.

\begin{centering}
\begin{table}[h!]
\newcolumntype{W}{D{.}{.}{2.3}}
\begin{adjustwidth}{-1cm}{}
\vspace{10pt}
\begin{tabular}{cccccccccc}
\toprule %[-2.08em]
\multicolumn{6}{c}{$|\overline{E}_{\Sigma_y}(z)|$} &
\multicolumn{4}{c}{$|\overline{E}_{\Sigma_z}(z)|$} \\[5 pt]
\multicolumn{2}{c}{$\theta = 0^\circ$} & 
\multicolumn{2}{c}{$\theta = 30^\circ$} & 
\multicolumn{2}{c}{$\theta = 60^\circ$} &
\multicolumn{2}{c}{$\theta = 30^\circ$} & 
\multicolumn{2}{c}{$\theta = 60^\circ$} \\
$z$,~mm & $\sqrt{\frac{\alpha}{\alpha_{\max}}}$ & 
$z$,~mm & $\sqrt{\frac{\alpha}{\alpha_{\max}}}$ & 
$z$,~mm & $\sqrt{\frac{\alpha}{\alpha_{\max}}}$ & 
$z$,~mm & $\sqrt{\frac{\alpha}{\alpha_{\max}}}$ & 
$z$,~mm & $\sqrt{\frac{\alpha}{\alpha_{\max}}}$ \\
 \midrule
20  &  0.415228  &  28  &  0.803219  &  14  &  0.395285  & 19  &  0.723747  &  46  &  0.748331  \\
22  &  0.473432  &  30  &  0.803219  &  16  &  0.395285  & 21  &  0.806718  &  48  &  0.83666   \\
24  &  0.776819  &  32  &  0.879883  &  18  &  0.395285  & 23  &  0.908514  &  50  &  0.894427  \\
26  &  0.946864  &  34  &  0.950382  &  20  &  0.467707  & 25  &  0.959497  &  52  &  0.959166  \\
28  &  1         &  36  &  0.983739  &  22  &  0.572822  & 27  &  0.992032  &  54  &  0.989949  \\
30  &  0.982607  &  38  &  1         &  24  &  0.684653  & 29  &  1         &  56  &  1         \\
32  &  0.890563  &  40  &  0.915811  &  26  &  0.790569  & 31  &  0.959497  &  58  &  1         \\
34  &  0.719195  &  42  &  0.861356  &  28  &  0.866025  & 33  &  0.899736  &  60  &  0.959166  \\
36  &  0.508548  &  44  &  0.803219  &  30  &  0.935414  & 35  &  0.835711  &  62  &  0.938083  \\
38  &  0.473432  &  -   &  -         &  32  &  0.976281  & 37  &  0.786796  &  64  &  0.883176  \\
-   &  -         &  -   &  -         &  34  &  1         & 39  &  0.776643  &  66  &  0.812404  \\
-   &  -         &  -   &  -         &  36  &  0.992157  & -   &  -         &  68  &  0.761577  \\
-   &  -         &  -   &  -         &  38  &  0.935414  & -   &  -         &  70  &  0.678233  \\
-   &  -         &  -   &  -         &  40  &  0.829156  & -   &  -         &  72  &  0.616441  \\
-   &  -         &  -   &  -         &  42  &  0.684653  & -   &  -         &  74  &  0.565685  \\
-   &  -         &  -   &  -         &  44  &  0.559017  & -   &  -         &  76  &  0.547723  \\
-   &  -         &  -   &  -         &  46  &  0.450694  & -   &  -         &  78  &  0.547723  \\
-   &  -         &  -   &  -         &  48  &  0.395285  & -   &  -         &  -   &  -         \\
-   &  -         &  -   &  -         &  50  &  0.353553  & -   &  -         &  -   &  -         \\
\bottomrule
\end{tabular}
\vspace{5 pt}
\caption{Зависимости составляющих $|\overline{E}_{\Sigma_y}(z)|$ и $|\overline{E}_{\Sigma_z}(z)|$} 
\end{adjustwidth}
\label{tab:tab1}
\end{table}
\end{centering}

\begin{centering}
\begin{table}[h!]
\newcolumntype{W}{D{.}{.}{2.3}}
%\begin{adjustwidth}{-1.5cm}{}
\vspace{10pt}
\begin{tabular}{cccccccc}
\toprule %[-2.08em]
\multicolumn{4}{c}{$|\overline{E}_{\Sigma_y}(y)|$} &
\multicolumn{4}{c}{$|\overline{E}_{\Sigma_z}(y)|$} \\[5 pt]
\multicolumn{2}{c}{$\theta = 30^\circ$} & 
\multicolumn{2}{c}{$\theta = 60^\circ$} &
\multicolumn{2}{c}{$\theta = 30^\circ$} & 
\multicolumn{2}{c}{$\theta = 60^\circ$} \\
$z$,~mm & $\sqrt{\frac{\alpha}{\alpha_{\max}}}$ &
$z$,~mm & $\sqrt{\frac{\alpha}{\alpha_{\max}}}$ &
$z$,~mm & $\sqrt{\frac{\alpha}{\alpha_{\max}}}$ &
$z$,~mm & $\sqrt{\frac{\alpha}{\alpha_{\max}}}$ \\
\midrule
14  &  0.62361   &  22  &  0.606977  &  34  &  0.680414  &  18  &  0.570088  \\
16  &  0.62361   &  24  &  0.675381  &  36  &  0.732828  &  20  &  0.591608  \\
18  &  0.65263   &  26  &  0.749269  &  38  &  0.816497  &  22  &  0.724569  \\
20  &  0.827759  &  28  &  0.858395  &  40  &  0.892354  &  24  &  0.758288  \\
22  &  0.922958  &  30  &  0.945905  &  42  &  0.952579  &  26  &  0.935414  \\
24  &  0.981307  &  32  &  1         &  44  &  0.981307  &  28  &  1         \\
26  &  1         &  34  &  0.955134  &  46  &  1         &  30  &  1         \\
28  &  0.981307  &  36  &  0.783604  &  48  &  0.981307  &  32  &  0.894427  \\
30  &  0.902671  &  38  &  0.648886  &  50  &  0.952579  &  34  &  0.758288  \\
32  &  0.816497  &  40  &  0.606977  &  52  &  0.892354  &  36  &  0.632456  \\
34  &  0.693888  &  -   &  -         &  54  &  0.805076  &  38  &  0.591608  \\
36  &  0.638284  &  -   &  -         &  56  &  0.745356  &  -   &  -         \\
38  &  0.62361   &  -   &  -         &  58  &  0.720083  &  -   &  -         \\
-   &  -         &  -   &  -         &  60  &  0.680414  &  -   &  -         \\
\bottomrule
\end{tabular}
\vspace{5 pt}
\caption{Зависимости составляющих $|\overline{E}_{\Sigma_y}(y)|$ и $|\overline{E}_{\Sigma_z}(y)|$} 
%\end{adjustwidth}
\label{tab:tab2}
\end{table}
\end{centering}

\subsection{Графики и рисунки}
Наиболее наглядным способом демонстрации и анализа поляризованных волн являются
поляризационные диаграммы. На рисунках \ref{fig:plot1} и \ref{fig:plot2}
представлены диаграммы для линейно и эллиптически поляризованных волн.
Они хорошо согласуются с теоретическими формами кривых, что
подтверждает корректность проведенных измерений и обработки их результатов.\\

\begin{figure}[hb!]
    \begin{center}
        \includegraphics[width=\textwidth]{plot1.pdf}
    \end{center}
    \vspace {-20 pt}
    \caption{График зависимости составляющих $|\overline{E}_{\Sigma_y}(z)|$}
    \label{fig:plot1}
\end{figure}

\begin{figure}[hb!]
    \begin{center}
        \includegraphics[width=\textwidth]{plot2.pdf}
    \end{center}
    \caption{График зависимости составляющих $|\overline{E}_{\Sigma_z}(z)|$}
    \label{fig:plot2}
\end{figure}

\begin{figure}[hb!]
    \begin{center}
        \includegraphics[width=\textwidth]{plot3.pdf}
    \end{center}
    \vspace {-20 pt}
    \caption{График зависимости составляющих $|\overline{E}_{\Sigma_y}(y)|$}
    \label{fig:plot3}
\end{figure}

\begin{figure}[h!]
    \begin{center}
        \includegraphics[width=\textwidth]{plot4.pdf}
    \end{center}
    \vspace {-20 pt}
    \caption{График зависимости составляющих $|\overline{E}_{\Sigma_z}(y)|$}
    \label{fig:plot4}
\end{figure}
\end{document}
